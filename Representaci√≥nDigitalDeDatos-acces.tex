\documentclass[spanish,A4,]{article}
\usepackage{sans}
\usepackage{amssymb,amsmath}
\usepackage{ifxetex,ifluatex}
\usepackage{fixltx2e} % provides \textsubscript
\ifnum 0\ifxetex 1\fi\ifluatex 1\fi=0 % if pdftex
  \usepackage[T1]{fontenc}
  \usepackage[utf8]{inputenc}
\else % if luatex or xelatex
  \ifxetex
    \usepackage{mathspec}
    \usepackage{xltxtra,xunicode}
  \else
    \usepackage{fontspec}
  \fi
  \defaultfontfeatures{Mapping=tex-text,Scale=MatchLowercase}
  \newcommand{\euro}{€}
\fi
% use upquote if available, for straight quotes in verbatim environments
\IfFileExists{upquote.sty}{\usepackage{upquote}}{}
% use microtype if available
\IfFileExists{microtype.sty}{\usepackage{microtype}}{}
\ifxetex
  \usepackage[setpagesize=false, % page size defined by xetex
              unicode=false, % unicode breaks when used with xetex
              xetex]{hyperref}
\else
  \usepackage[unicode=true]{hyperref}
\fi
\hypersetup{breaklinks=true,
            bookmarks=true,
            pdfauthor={},
            pdftitle={},
            colorlinks=true,
            citecolor=blue,
            urlcolor=blue,
            linkcolor=magenta,
            pdfborder={0 0 0}}
\urlstyle{same}  % don't use monospace font for urls
\setlength{\parindent}{0pt}
\setlength{\parskip}{6pt plus 2pt minus 1pt}
\setlength{\emergencystretch}{3em}  % prevent overfull lines
\setcounter{secnumdepth}{0}
\ifxetex
  \usepackage{polyglossia}
  \setmainlanguage{spanish}
\else
  \usepackage[spanish]{babel}
\fi

\title{Representación de la información}

\begin{document}
\maketitle

\section{Representación de la
Información}\label{representaciuxf3n-de-la-informaciuxf3n}

Veremos de qué manera puede ser tratada mediante computadoras la
información correspondiente a números, textos, imágenes y otros datos.
Necesitaremos conocer las formas de representación de datos, y
comenzaremos por los datos numéricos.

\subsection{Representación de datos
numéricos}\label{representaciuxf3n-de-datos-numuxe9ricos}

Hemos visto ejemplos de sistemas de numeración: en base 6, en base 10, o
decimal, en base 2, o binario, en base 16, o hexadecimal, y en base 8, u
octal; y sabemos convertir la representación de un número en cada una de
estas bases, a los sistemas en las demás bases. Sin embargo, aún nos
falta considerar la representación numérica de varios casos importantes:

\begin{itemize}
\itemsep1pt\parskip0pt\parsep0pt
\item
  Hemos utilizado estos sistemas para representar únicamente números
  \textbf{enteros}. Nos falta ver de qué manera representar números
  racionales, es decir aquellos que tienen una parte fraccionaria (los
  ``decimales'').
\item
  Además estos enteros han sido siempre \textbf{no negativos}, es decir,
  sabemos representar únicamente el 0 y los naturales. Nos falta
  considerar los negativos.
\item
  Por otra parte, no nos hemos planteado el problema de la
  \textbf{cantidad de dígitos}. Idealmente, un sistema de numeración
  puede usar infinitos dígitos para representar números arbitrariamente
  grandes. Si bien esto es matemáticamente correcto, las computadoras
  son objetos físicos que tienen unas ciertas limitaciones, y con ellas
  no es posible representar números de infinita cantidad de dígitos.
\end{itemize}

En esta parte de la unidad mostraremos sistemas de representación
utilizados en computación que permiten tratar estos problemas.

\subsubsection{Clasificación de los
números}\label{clasificaciuxf3n-de-los-nuxfameros}

Es conveniente repasar la clasificación de los diferentes conjuntos de
números y conocer las diferencias importantes entre éstos. Los títulos
en el cuadro (tomado de Wikipedia) son referencias a los artículos
enciclopédicos sobre cada uno de esos conjuntos.

\begin{itemize}
\itemsep1pt\parskip0pt\parsep0pt
\item
  \href{https://es.m.wikipedia.org/wiki/N\%C3\%BAmero sub complejo}{Números
  complejos}
\item
  \href{https://es.m.wikipedia.org/wiki/N\%C3\%BAmero sub complejo}{Complejos}
\item
  \href{https://es.m.wikipedia.org/wiki/N\%C3\%BAmero sub real}{Reales}
\item
  \href{https://es.m.wikipedia.org/wiki/N\%C3\%BAmero sub racional}{Racionales}
\item
  \href{https://es.m.wikipedia.org/wiki/N\%C3\%BAmero sub entero}{Enteros}
\item
  \href{https://es.m.wikipedia.org/wiki/N\%C3\%BAmero sub natural}{Naturales}
\item
  \href{https://es.m.wikipedia.org/wiki/Uno}{Uno}
\item
  \href{https://es.m.wikipedia.org/wiki/N\%C3\%BAmero sub primo}{Naturales
  primos}
\item
  \href{https://es.m.wikipedia.org/wiki/N\%C3\%BAmero sub compuesto}{Naturales
  compuestos}
\item
  \href{https://es.m.wikipedia.org/wiki/Cero}{Cero}
\item
  \href{https://es.m.wikipedia.org/wiki/Entero sub negativo}{Enteros
  negativos}
\item
  \href{https://es.m.wikipedia.org/wiki/N\%C3\%BAmero sub fraccionario}{Fraccionarios}
\item
  \href{https://es.m.wikipedia.org/wiki/Fracci\%C3\%B3n sub propia}{Fracción
  propia}
\item
  \href{https://es.m.wikipedia.org/wiki/Fracci\%C3\%B3n sub impropia}{Fracción
  impropia}
\item
  \href{https://es.m.wikipedia.org/wiki/N\%C3\%BAmero sub irracional}{Irracionales}
\item
  \href{https://es.m.wikipedia.org/wiki/N\%C3\%BAmero sub algebraico}{Irracionales
  algebraicos}
\item
  \href{https://es.m.wikipedia.org/wiki/N\%C3\%BAmero sub trascendente}{Trascendentes}
\item
  \href{https://es.m.wikipedia.org/wiki/N\%C3\%BAmero sub imaginario}{Imaginarios}
\end{itemize}

\textbf{Preguntas}

\begin{itemize}
\itemsep1pt\parskip0pt\parsep0pt
\item
  El \textbf{cero}, ¿es un natural?
\item
  ¿Existen números naturales negativos? ¿Y racionales negativos?
\item
  ¿Es correcto decir que un racional tiene una parte decimal que es, o
  bien finita, o bien periódica?
\item
  ¿Puede haber dos expresiones diferentes para el mismo número, en el
  mismo sistema de numeración decimal?
\item
  El número 0.9999\ldots{} con 9 periódico, y el número 1, ¿son dos
  números diferentes o el mismo número? Si son diferentes, ¿qué número
  se encuentra entre ellos dos?
\item
  El número 1 es a la vez natural y entero. ¿Por qué no puede haber un
  número que sea a la vez racional e irracional?
\item
  ¿Por qué jamás podremos computar la sucesión completa de decimales de
  $PI$?
\end{itemize}

\subsubsection{Datos enteros}\label{datos-enteros}

Veremos un sistema de representación de datos no negativos, llamado
\textbf{sin signo}, y tres sistemas de representación de datos numéricos
enteros, llamados \textbf{signo-magnitud}, \textbf{complemento a 2} y
\textbf{notación en exceso}.

\subsubsection{Datos fraccionarios}\label{datos-fraccionarios}

Para representar fraccionarios consideraremos los sistemas de
\textbf{punto fijo} y \textbf{punto flotante}.

\subsection{Rango de representación}\label{rango-de-representaciuxf3n}

Cada sistema de representación de datos numéricos tiene su propio
\textbf{rango de representación} (que podemos abreviar \textbf{RR}), o
intervalo de números representables. Ningún número fuera de este rango
puede ser representado en dicho sistema. Conocer este intervalo es
importante para saber con qué limitaciones puede enfrentarse un programa
que utilice alguno de esos sistemas.

El rango de los números representados bajo un sistema está dado por sus
\textbf{límites inferior y superior}, que definen qué zona de la recta
numérica puede ser representada. Como ocurre con todo intervalo numérico
cerrado, el rango de representación puede ser escrito como $[a, b]$,
donde $a$ y $b$ son sus límites inferior y superior, respectivamente.

Por la forma en que están diseñados, algunos sistemas de representación
sólo pueden representar números muy pequeños, o sólo positivos, o tanto
negativos como positivos. En general, el RR \textbf{será más grande
cuantos más dígitos binarios}, o bits, tenga el sistema. Sin embargo, el
RR depende también de la forma como el sistema \textbf{utilice} esos
dígitos binarios, ya que un sistema puede ser más o menos
\textbf{eficiente} que otro en el uso de esos dígitos, aunque la
cantidad de dígitos sea la misma en ambos sistemas.

Por lo tanto, decimos que el rango de representación depende a la vez de
la \textbf{cantidad de dígitos} y de la \textbf{forma de funcionamiento}
del sistema de representación.

\subsection{Representación sin signo
SS(k)}\label{representaciuxf3n-sin-signo-ssk}

Consideremos primero qué ocurre cuando queremos representar números
enteros \textbf{no negativos} (es decir, \textbf{positivos o cero})
sobre una cantidad fija de bits.

En el sistema \textbf{sin signo}, simplemente usamos el sistema binario
de numeración, tal como lo conocemos, \textbf{pero limitándonos a una
cantidad fija} de dígitos binarios o bits. Podemos entonces abreviar el
nombre de este sistema como \textbf{SS(k)}, donde $k$ es la cantidad
fija de bits, o ancho, de cada número representado.

\subsubsection{Rango de representación de
SS(k)}\label{rango-de-representaciuxf3n-de-ssk}

¿Cuál será el rango de representación? El \textbf{cero} puede
representarse, así que el límite inferior del rango de representación
será 0. Pero ¿cuál será el límite superior? Es decir, si la cantidad de
dígitos binarios en este sistema es $k$, ¿cuál es el número más grande
que podremos representar?

Podemos estudiarlo de dos maneras.

\textbf{1. Usando combinatoria}

Contemos cuántos números diferentes podemos escribir con $k$ dígitos
binarios. Imaginemos un número binario cualquiera con $k$ dígitos. El
dígito de más a la derecha tiene únicamente dos posibilidades (0 o 1).
Por cada una de éstas hay nuevamente dos posibilidades para el siguiente
hacia la izquierda (lo que da las cuatro posibilidades 00, 01, 10, 11).
Por cada una de éstas, hay dos posibilidades para el siguiente (dando
las ocho posibilidades 0 0 0, 0 0 1, 0 1 0, 0 1 1, 1 0 0, 1 0 1, 1 1 0, 1 1 1), etc., y
así hasta la posición $k$. No hay más posibilidades. Como hemos
multiplicado 2 por sí mismo $k$ veces, la cantidad de números que se
pueden escribir es $2 elevado a la k$. Luego, el número más grande posible es
$2 elevado a la k - 1$. (\textbf{Pregunta}: ¿Por qué $2 elevado a la k - 1$ y no $2 elevado a la k$?).

\textbf{2. Usando álgebra}

El número más grande que podemos representar en un sistema sin signo a
$k$ dígitos es, seguramente, aquel donde todos los $k$ dígitos valen
\textbf{1}. La Expresión General que hemos visto nos dice que si un
número $n$ está escrito en base 2, \textbf{con $k$ dígitos}, entonces
\[n = x sub k-1  por  2 elevado a la {k-1} + ... + x sub 1 por 2 elevado a la 1+x sub 0 por 2 elevado a la 0\] y, si
queremos escribir el más grande de todos, deberán ser todos los $x sub i$
iguales a 1. (\textbf{Pregunta}: ¿Por qué si el número $n$ tiene $k$
dígitos binarios, el índice del más significativo es $k-1$ y no $k$?)

Esta suma vale entonces

\[ x sub k-1  por  2 elevado a la {k-1} + ... + x sub 1 por 2 elevado a la 1+x sub 0 por 2 elevado a la 0 = \]
\[  = 1 por  2 elevado a la {k-1} + ... + 1 por 2 elevado a la 1+1 por 2 elevado a la 0 = \]
\[  = 2 elevado a la {k-1} + ... + 2 elevado a la 1+2 elevado a la 0 = \] \[  = 2 elevado a la {k}-1 \]

Usando ambos argumentos hemos llegado a que el número más grande que
podemos representar con $k$ dígitos binarios es $2 elevado a la k-1$. Por lo tanto,
\textbf{el rango de representación de un sistema sin signo a $k$
dígitos, o SS(k), es $[0, 2 elevado a la k - 1]$}. Todos los números representables
en esta clase de sistemas son \textbf{positivos o cero}.

\textbf{Ejemplos}

\begin{itemize}
\itemsep1pt\parskip0pt\parsep0pt
\item
  Para un sistema de representación sin signo a 8 bits:
  $[0, 2 elevado a la 8-1] = [0, 255]$
\item
  Con 16 bits: $[0, 2 elevado a la {16}-1] = [0, 65.535]$
\item
  Con 32 bits: $[0, 2 elevado a la {32}-1] = [0, 4.294.967.295]$
\end{itemize}

\subsection{Representación con signo}\label{representaciuxf3n-con-signo}

En la vida diaria manejamos continuamente números negativos, y los
distinguimos de los positivos simplemente agregando un signo ``menos''.
Representar esos datos en la memoria de la computadora no es tan
directo, porque, como hemos visto, la memoria \textbf{solamente puede
alojar ceros y unos}. Es decir, ¡no podemos simplemente guardar un signo
``menos''! Lo único que podemos hacer es almacenar secuencias de ceros y
unos.

Esto no era un problema cuando los números eran no negativos. Para poder
representar, ahora, tanto números \textbf{positivos como negativos},
necesitamos cambiar la forma de representación. Esto quiere decir que
una secuencia particular de dígitos binarios, que en un sistema sin
signo tiene un cierto significado, ahora tendrá un significado
diferente. Algunas secuencias, que antes representaban números
positivos, ahora representarán negativos.

Veremos los \textbf{sistemas de representación con signo} llamados
\textbf{Signo-magnitud (SM)}, \textbf{Complemento a 2 (C2)} y
\textbf{Notación en exceso}.

Es importante tener en cuenta que \textbf{solamente se puede operar
entre datos representados con el mismo sistema de representación}, y que
el \textbf{resultado} de toda operación \textbf{vuelve a estar
representado en el mismo sistema}.

\textbf{Preguntas}

\begin{itemize}
\itemsep1pt\parskip0pt\parsep0pt
\item
  ¿Cuáles son los límites del rango de representación de un sistema de
  representación numérica?
\item
  Un número escrito en un sistema de representación \textbf{con signo},
  ¿es siempre negativo?
\item
  ¿Para qué querríamos escribir un número positivo en un sistema de
  representación con signo?
\end{itemize}

\subsection{Sistema de Signo-magnitud
SM(k)}\label{sistema-de-signo-magnitud-smk}

El sistema de \textbf{Signo-magnitud} no es el más utilizado en la
práctica, pero es el más sencillo de comprender. Se trata simplemente de
utilizar un bit (el de más a la izquierda) para representar el
\textbf{signo}. Si este bit tiene valor 0, el número representado es
positivo; si es 1, es negativo. Los demás bits se utilizan para
representar la \textbf{magnitud}, es decir, el \textbf{valor absoluto}
del número en cuestión.

\textbf{Ejemplos}

\begin{itemize}
\itemsep1pt\parskip0pt\parsep0pt
\item
  $7 en base 10 = 0 0 0 0 0 1 1 1 en base 2$
\item
  $-7 en base 10 = 1 0 0 0 0 1 1 1 en base 2$
\end{itemize}

Como estamos reservando un bit para expresar el signo, ese bit ya no se
puede usar para representar magnitud; y como el sistema tiene una
cantidad de bits fija, el RR ya no podrá representar el número máximo
que era posible con el sistema \textbf{sin signo}.

\subsubsection{Rango de representación de
SM(k)}\label{rango-de-representaciuxf3n-de-smk}

\begin{itemize}
\itemsep1pt\parskip0pt\parsep0pt
\item
  En todo número escrito en el sistema de signo-magnitud a $k$ bits, ya
  sea positivo o negativo, hay un bit reservado para el signo, lo que
  implica que quedan $k-1$ bits para representar su valor absoluto.
\item
  Siendo un valor absoluto, estos $k-1$ bits representan un número
  \textbf{no negativo}. Además este número está representado con el
  sistema \textbf{sin signo} sobre $k-1$ bits, es decir, SS(k-1).
\item
  Este número no negativo en SS(k-1) tendrá un valor máximo
  representable que coincide con el \textbf{límite superior} del rango
  de representación \textbf{de SS(k-1)}.
\item
  Sabemos que el rango de representación de SS(k) es $[0, 2 elevado a la k-1]$. Por
  lo tanto, el rango de SS(k-1), reemplazando, será $[0, 2 elevado a la {k-1} -1]$.
\item
  Esto quiere decir que el número representable en SM(k) cuyo valor
  absoluto es máximo, es $2 elevado a la {k-1}-1$. Por lo tanto éste es el límite
  superior del rango de representación de SM(k).
\item
  Pero en SM(k) también se puede representar su opuesto negativo,
  simplemente cambiando el bit más alto por 1. El opuesto del máximo
  positivo representable es a su vez el número más pequeño, negativo,
  representable: $-(2 elevado a la {k-1}-1)$.
\end{itemize}

Con lo cual hemos calculado tanto el límite inferior como el superior
del rango de representación de SM(k), que, finalmente, es
$[-(2 elevado a la {k-1}-1),2 elevado a la {k-1}-1]$.

\subsubsection{Limitaciones de
Signo-Magnitud}\label{limitaciones-de-signo-magnitud}

Si bien \textbf{SM(k)} es simple, no es tan efectivo, por varias
razones:

\begin{itemize}
\itemsep1pt\parskip0pt\parsep0pt
\item
  Existen dos representaciones del 0 (``positiva'' y ``negativa''), lo
  cual desperdicia un representante.
\item
  Esto acorta el rango de representación.
\item
  La aritmética en SM no es fácil, ya que cada operación debe comenzar
  por averiguar si los operandos son positivos o negativos, operar con
  los valores absolutos y ajustar el resultado de acuerdo al signo
  reconocido anteriormente.
\item
  El problema aritmético se agrava con la existencia de las dos
  representaciones del cero: cada vez que un programa quisiera comparar
  un valor resultado de un cómputo con 0, debería hacer \textbf{dos}
  comparaciones.
\end{itemize}

Por estos motivos, el sistema de SM dejó de usarse y se diseñó un
sistema que eliminó estos problemas, el sistema de \textbf{complemento a
2}.

\subsection{Sistema de Complemento a
2}\label{sistema-de-complemento-a-2}

Para comprender el sistema de complemento a 2 es necesario primero
conocer la \textbf{operación} de complementar a 2.

\subsubsection{Operación de Complemento a
2}\label{operaciuxf3n-de-complemento-a-2}

La \textbf{operación} de complementar a 2 consiste aritméticamente en
obtener el \textbf{opuesto} de un número (el que tiene el mismo valor
absoluto pero signo opuesto).

Para obtener el complemento a 2 de un número escrito en base 2,
\textbf{se invierte cada uno de los bits (reemplazando 0 por 1 y
viceversa) y al resultado se le suma 1}.

\textbf{Otra forma}

Otro modo de calcular el complemento a 2 de un número en base 2 es
\textbf{copiar los bits, desde la derecha, hasta el primer 1 inclusive;
e invertir todos los demás a la izquierda}.

\textbf{Propiedad fundamental}

El resultado de esta operación, C2($a$), es el opuesto del número
original $a$, y por lo tanto tiene la propiedad de que $a$ y C2($a$)
suman 0:

\[C2(a) + a = 0\]

\textbf{Comprobación}

Podemos comprobar si la complementación fue bien hecha aplicando la
\textbf{propiedad fundamental} del complemento. Si, al sumar nuestro
resultado con el número original, no obtenemos 0, corresponde revisar la
operación.

\textbf{Ejemplos}

\begin{itemize}
\itemsep1pt\parskip0pt\parsep0pt
\item
  Busquemos el complemento a 2 de $1 1 1 0 1 0$. Invirtiendo todos los bits,
  obtenemos $0 0 0 1 0 1$. Sumando 1, queda $0 0 0 1 1 0$.
\item
  Busquemos el complemento a 2 de $0 0 1 1$. Invirtiendo todos los bits,
  obtenemos $1 1 0 0$. Sumando 1, queda $1 1 0 1$.
\item
  Comprobemos que el resultado obtenido en el último caso, $1 1 0 1$, es
  efectivamente el opuesto de $0 0 1 1$: $0 0 1 1 + 1 1 0 1 = 0$.
\end{itemize}

\subsubsection{Representación en Complemento a
2}\label{representaciuxf3n-en-complemento-a-2}

Ahora que contamos con la \textbf{operación de complementar a 2},
podemos ver cómo se construye el \textbf{sistema de representación en
Complemento a 2}.

Para representar un número $a$ en complemento a 2 a k bits, comenzamos
por considerar su signo:

\begin{itemize}
\itemsep1pt\parskip0pt\parsep0pt
\item
  Si $a$ es positivo o cero, lo representamos como en SM(k), es decir,
  lo escribimos en base 2 a k bits.
\item
  Si $a$ es negativo, tomamos su valor absoluto y lo complementamos a 2.
\end{itemize}

\textbf{Ejemplos}

\begin{itemize}
\itemsep1pt\parskip0pt\parsep0pt
\item
  Representemos el número 17 en complemento a 2 con 8 bits. Como es
  positivo, lo escribimos en base 2, obteniendo $0 0 0 1 0 0 0 1$, que es 17 en
  notación complemento a 2 con 8 bits.
\item
  Representemos el número -17 en complemento a 2 con 8 bits. Como es
  negativo, escribimos su valor absoluto en base 2, que es $0 0 0 1 0 0 0 1$, y
  lo complementamos a 2. El resultado final es $1 1 1 0 1 1 1 1$ que es -17 en
  notación complemento a 2 con 8 bits.
\end{itemize}

\subsubsection{Conversión de C2 a base
10}\label{conversiuxf3n-de-c2-a-base-10}

Para convertir un número $n$, escrito en el sistema de complemento a 2,
a decimal, lo primero es determinar el signo. Si el bit más alto es 1,
$n$ es negativo. En otro caso, $n$ es positivo. Utilizaremos esta
información enseguida.

\begin{itemize}
\itemsep1pt\parskip0pt\parsep0pt
\item
  Si $n$ es positivo, se interpreta el número como en el sistema sin
  signo, es decir, se utiliza la Expresión General para hacer la
  conversión de base como normalmente.
\item
  Si $n$ es negativo, se lo complementa a 2, obteniendo el opuesto de
  $n$. Este número, que ahora es positivo, se convierte a base 10 como
  en el caso anterior; y finalmente se le agrega el signo ``-'' para
  reflejar el hecho de que es negativo.
\end{itemize}

\textbf{Ejemplos}

\begin{itemize}
\itemsep1pt\parskip0pt\parsep0pt
\item
  Convertir a decimal $n = 0 0 0 1 0 0 0 1$. Es positivo, luego, aplicamos la
  Expresión General dando $17 en base 10$.
\item
  Convertir a decimal $n = 1 1 1 0 1 1 1 1$. Es negativo; luego, lo
  complementamos a 2 obteniendo $0 0 0 1 0 0 0 1$. Aplicamos la Expresión
  General obteniendo $17 en base 10$. Como $n$ era negativo, agregamos el
  signo menos y obtenemos el resultado final $-17 en base 10$.
\end{itemize}

\subsubsection{RR de C2 con $k$ bits}\label{rr-de-c2-con-k-bits}

La forma de utilizar los bits en el sistema de complemento a 2 permite
recuperar un representante que estaba desperdiciado en SM.

El rango de representación del sistema complemento a 2 sobre $k$ bits es
$[-(2 elevado a la {k-1}), 2 elevado a la {k-1}-1]$. El límite superior del RR de C2 es el mismo
que el de SM, pero el \textbf{límite inferior} es menor; luego el RR de
C2 es mayor que el de SM.

El sistema de complemento a 2 tiene otras ventajas sobre SM:

\begin{itemize}
\itemsep1pt\parskip0pt\parsep0pt
\item
  El cero tiene una única representación, lo que facilita las
  comparaciones.
\item
  Las cuentas se hacen bit a bit, en lugar de requerir comprobaciones de
  signo.
\item
  El mecanismo de cálculo es eficiente y fácil de implementar en
  hardware.
\item
  Solamente se requiere diseñar un algoritmo para \textbf{sumar}, no uno
  para sumar y otro para restar.
\end{itemize}

\subsubsection{Comparando rangos de
representación}\label{comparando-rangos-de-representaciuxf3n}

Diferentes sistemas, entonces, tienen diferentes rangos de
representación. Si construimos un cuadro donde podamos comparar los
rangos de representación \textbf{sin signo, signo-magnitud y complemento
a 2} para una misma cantidad de bits, veremos que todas las
combinaciones de bits están utilizadas, sólo que de diferente forma.

El cuadro comparativo para cuatro bits mostrará que las combinaciones
0 0 0 0\ldots{}1 1 1 1 representan los primeros 16 números no negativos para
el sistema sin signo, mientras que esas mismas combinaciones tienen otro
significado en los sistemas con signo. En éstos últimos, una misma
combinación con el bit más significativo en 1 siempre es negativa, pero
el orden en que aparecen esas combinaciones es diferente entre SM y C2.

Por otro lado, los números positivos quedan representados por
combinaciones idénticas en los tres sistemas, hasta donde lo permite el
rango de representación de cada uno.

Si descartamos el bit de signo y consideramos sólo las magnitudes, los
números negativos en SM aparecen con sus magnitudes crecientes
alejándose del 0, mientras que en C2 esas magnitudes comienzan en cero
al representar el negativo más pequeño posible y crecen a medida que se
acercan al cero.

\subsubsection{Aritmética en C2}\label{aritmuxe9tica-en-c2}

Una gran ventaja que aporta el sistema en Complemento a 2 es que los
diseñadores de hardware no necesitan implementar algoritmos de resta
además de los de la suma. Cuando se necesita efectuar una resta,
\textbf{se complementa el sustraendo} y luego se lo \textbf{suma} al
minuendo. Las computadoras no restan: siempre suman.

Por ejemplo, la operación $9 - 8$ se realiza como $9 + (-8)$, donde (-8)
es el complemento a 2 de 8.

\textbf{Preguntas}

\begin{itemize}
\itemsep1pt\parskip0pt\parsep0pt
\item
  Un número en complemento a 2, ¿tiene siempre su bit más a la izquierda
  en 1?
\item
  El complemento a 2 de un número, es decir, \textbf{C2(x)}, ¿es siempre
  un número negativo?
\item
  ¿Quién es \textbf{C2(0)}?
\item
  ¿Cuánto vale \textbf{C2(C2(x))}? Es decir, ¿qué pasa si complemento a
  2 el complemento a 2 de $x$?
\item
  ¿Cuánto vale \textbf{x + C2(x)}? Es decir, ¿qué pasa si sumo a $x$ su
  propio complemento a 2?
\item
  ¿Cómo puedo verificar si calculé correctamente un complemento a 2?
\end{itemize}

\subsubsection{Overflow o desbordamiento en
C2}\label{overflow-o-desbordamiento-en-c2}

En todo sistema de ancho fijo, la suma de \textbf{dos números positivos,
o de dos números negativos} puede dar un resultado que sea imposible de
representar debido a las limitaciones del rango de representación. Este
problema se conoce como desbordamiento, u \emph{overflow}. Cuando ocurre
una situación de overflow, el resultado de la operación \textbf{no es
válido} y debe ser descartado.

Si conocemos los valores en decimal de dos números que queremos sumar,
usando nuestro conocimiento del rango de representación del sistema
podemos saber si el resultado quedará dentro de ese rango, y así
sabemos, de antemano, si ese resultado será válido. Pero las
computadoras no tienen forma de conocer a priori esta condición, ya que
todo lo que tienen es la representación en C2 de ambos números. Por eso
necesitan alguna forma de detectar las situaciones de overflow, y el
modo más fácil para ellas es comprobar los dos últimos bits de la fila
de bits de acarreo o \emph{carry}.

El último bit de la fila de carry, el que se posiciona en la última de
las $k$ columnas de la representación, se llama \emph{carry-in}. El
siguiente bit de carry, el que ya no puede acarrearse sobre ningún
dígito válido porque se han rebasado los $k$ dígitos de la
representación, se llama el \emph{carry-out}.

\begin{itemize}
\itemsep1pt\parskip0pt\parsep0pt
\item
  Si, luego de efectuar una suma en C2, los valores de los bits de
  \emph{carry-in} y \emph{carry-out} son \textbf{iguales}, entonces la
  computadora detecta que el resultado no ha desbordado y que \textbf{la
  suma es válida}. La operación de suma se ha efectuado exitosamente.
\item
  Si, luego de efectuar una suma en C2, los valores de los bits de
  \emph{carry-in} y \emph{carry-out} son \textbf{diferentes}, entonces
  la computadora detecta que el resultado ha desbordado y que \textbf{la
  suma no es válida}. La operación de suma no se ha llevado a cabo
  exitosamente, y el resultado debe ser descartado.
\end{itemize}

\textbf{Suma sin overflow}

Siguiendo atentamente la secuencia de bits de carry podemos detectar,
igual que lo hace la computadora, si se producirá un desbordamiento. En
el caso de la operación $23 + (-9)$, el resultado (que es 14) cae dentro
del rango de representación, y esto se refleja en los bits de
\emph{carry-in} y de \emph{carry-out}, cuyos valores son iguales.

\textbf{Suma con overflow}

En el caso de la operación $123 + 9$ en C2 a 8 bits, el resultado (que
es 132) cae fuera del rango de representación. Esto se refleja en los
bits de \emph{carry-in} y de \emph{carry-out}, que son diferentes. El
resultado no es válido y debe ser descartado.

\textbf{Preguntas}

\begin{itemize}
\itemsep1pt\parskip0pt\parsep0pt
\item
  ¿Qué condición sobre los bits de carry permite asegurar que \textbf{no
  habrá} overflow?
\item
  ¿Para qué sistemas de representación numérica usamos la condición de
  detección de overflow?
\item
  ¿Puede existir overflow al sumar dos números de diferente signo?
\item
  ¿Qué condición sobre los bits \textbf{de signo} de los operandos
  permite asegurar que \textbf{no habrá} overflow?
\item
  ¿Puede haber casos de overflow al sumar dos números negativos?
\item
  ¿Puede haber casos de overflow al restar dos números?
\end{itemize}

\subsubsection{Extensión de signo en
C2}\label{extensiuxf3n-de-signo-en-c2}

Para poder efectuar una suma de dos números, ambos operandos deben estar
representados en el mismo sistema de representación.

\begin{itemize}
\itemsep1pt\parskip0pt\parsep0pt
\item
  Una suma de dos operandos donde uno esté, por ejemplo, en SM y el otro
  en C2, no tiene sentido aritmético.
\item
  Además, la cantidad de bits de representación debe ser la misma.
\end{itemize}

En una suma en C2, si uno de los operandos estuviera expresado en un
sistema con menos bits que el otro, será necesario convertirlo al
sistema del otro (\textbf{extenderlo}) y operar con ambos en ese sistema
de mayor ancho.

Si el operando en el sistema de menor ancho es positivo, la extensión se
realiza simplemente \textbf{completando con ceros a la izquierda} hasta
obtener la cantidad de dígitos del otro sistema. Si el operando del
menor ancho es negativo, la extensión de signo se hace \textbf{agregando
unos}.

\textbf{Ejemplos}

\begin{itemize}
\itemsep1pt\parskip0pt\parsep0pt
\item
  $A + B = 0 0 1 0 1 0 1 1 en base 2 + 0 0 1 0 1 en base 2$

  \begin{itemize}
  \itemsep1pt\parskip0pt\parsep0pt
  \item
    A está en $C elevado a la 8 sub 2$ y B en $C elevado a la 5 sub 2$ → llevar ambos a $C elevado a la 8 sub 2$
  \item
    Se completa B (positivo) como $0 0 0 0 0 1 0 1 en base 2$
  \end{itemize}
\item
  $A + B = 1 0 1 0 en base 2 + 0 1 1 0 1 0 0 en base 2$

  \begin{itemize}
  \itemsep1pt\parskip0pt\parsep0pt
  \item
    A está en $C elevado a la 4 sub 2$ y B en $C elevado a la 7 sub 2$ → llevar ambos a $C elevado a la 7 sub 2$
  \item
    Se completa A (negativo) como $1 1 1 1 0 1 0 en base 2$
  \end{itemize}
\end{itemize}

\subsection{Notación en exceso o
\emph{bias}}\label{notaciuxf3n-en-exceso-o-bias}

En un sistema de notación en exceso, se elige un intervalo $[a, b]$ de
enteros a representar, y todos los valores dentro del intervalo se
representan con una secuencia de bits de la misma longitud.

La cantidad de bits deberá ser la necesaria para representar todos los
enteros del intervalo, inclusive los límites, y por lo tanto estará en
función de la longitud del intervalo. Un intervalo $[a, b]$ de enteros,
con sus límites incluidos, comprende exactamente $n = b - a + 1$
valores. Esta longitud del intervalo debe ser cubierta con una cantidad
$k$ de bits suficiente, lo cual obliga a que $2 elevado a la k \geq n$. Supongamos
que $n$ sea una potencia de 2 para facilitar las ideas, de forma que
$2 elevado a la k = n$.

Las $2 elevado a la k$ secuencias de $k$ bits, ordenadas como de costumbre según su
valor aritmético, se aplican a los enteros en $[a, b]$, uno por uno. Es
decir, si usamos 3 bits, las secuencias serán 0 0 0, 0 0 1, 0 1 0, \ldots{}
hasta 1 1 1; y los valores representados serán respectivamente:

\begin{itemize}
\itemsep1pt\parskip0pt\parsep0pt
\item
  0 0 0 = $a$
\item
  0 0 1 = $a + 1$
\item
  0 1 0 = $a + 2$
\item
  \ldots{}
\item
  1 1 1 = $b$
\end{itemize}

Notemos que tanto $a$ como $b$ pueden ser \textbf{negativos}. Así
podemos representar intervalos de enteros arbitrarios con secuencias de
$k$ bits, lo que nos vuelve a dar un sistema de representación con
signo.

Con este método no es necesario que el bit de orden más alto represente
el signo. Tampoco que el intervalo contenga la misma cantidad de números
negativos que positivos o cero, aunque para la mayoría de las
aplicaciones es lo más razonable.

El sistema en exceso se utiliza como componente de otro sistema de
representación más complejo, la representación en punto flotante.

\subsubsection{Conversión entre exceso y
decimal}\label{conversiuxf3n-entre-exceso-y-decimal}

Una vez establecido un sistema en exceso que representa el intervalo
$[a, b]$ en $k$ bits:

\begin{itemize}
\itemsep1pt\parskip0pt\parsep0pt
\item
  Para calcular la secuencia binaria que corresponde a un valor decimal
  $d$, a $d$ \textbf{le restamos} $a$ y luego convertimos el resultado
  (que será \textbf{no negativo}) a \textbf{SS(k)}, es decir, a binario
  sin signo sobre $k$ bits.
\item
  Para calcular el valor decimal $d$ representado por una secuencia
  binaria, convertimos la secuencia a decimal como en \textbf{SS(k)}, y
  al resultado (que será \textbf{no negativo}) le \textbf{sumamos} el
  valor de $a$.
\end{itemize}

\textbf{Ejemplos}

Representemos en sistema en exceso el intervalo $[10, 25]$ (que contiene
$25 - 10 + 1 = 16$ enteros). Como necesitamos 16 secuencias binarias,
usaremos 4 bits que producirán las secuencias 0 0 0 0, 0 0 0 1, \ldots{},
1 1 1 1.

\begin{itemize}
\itemsep1pt\parskip0pt\parsep0pt
\item
  Para calcular la secuencia que corresponde al número 20, hacemos
  $20 - 10 = 10$ y el resultado será la secuencia \textbf{1 0 1 0}.
\item
  Para calcular el valor decimal que está representando la secuencia
  \textbf{1 0 1 1}, convertimos 1 0 1 1 a decimal, que es 11, y le sumamos 10;
  el resultado es $21$.
\end{itemize}

Representemos en sistema en exceso el intervalo $[-3, 4]$ (que contiene
$4 -(-3) + 1 = 8$ enteros). Como necesitamos 8 secuencias binarias,
usaremos 3 bits que producirán las secuencias 0 0 0, 0 0 1, \ldots{}, 1 1 1.

\begin{itemize}
\itemsep1pt\parskip0pt\parsep0pt
\item
  Para calcular la secuencia que corresponde al número 2, hacemos
  $2 -(-3) = 5$ y el resultado será la secuencia \textbf{1 0 1}.
\item
  Para calcular el valor decimal que está representando la secuencia
  \textbf{0 1 1}, convertimos 0 1 1 a decimal, que es 3, y le sumamos -3; el
  resultado es $0$.
\end{itemize}

\textbf{Preguntas sobre Notación en Exceso}

\begin{itemize}
\itemsep1pt\parskip0pt\parsep0pt
\item
  Dado un valor decimal a representar, ¿cómo calculamos el binario?
\item
  Dado un binario, ¿cómo calculamos el valor decimal representado?
\item
  El sistema en exceso ¿destina un bit para representar el signo?
\item
  ¿Se puede representar un intervalo que no contenga el cero?
\item
  ¿Cómo se comparan dos números en exceso para saber cuál es el mayor?
\end{itemize}

\subsection{Representación de
fraccionarios}\label{representaciuxf3n-de-fraccionarios}

\subsubsection{Racionales}\label{racionales}

Los números fraccionarios son aquellos \textbf{racionales} que no son
enteros. Se escriben como una razón, fracción o cociente de dos enteros.
Por ejemplo, $3/4$ y $-12/5$ son números fraccionarios. El signo de
división que usamos para escribir las fracciones tiene precisamente ese
significado aritmético: si hacemos la operación de división
correspondiente entre numerador y divisor de la fracción, obtenemos la
forma decimal del mismo número, con \textbf{una parte entera y una parte
decimal}. Así, por ejemplo, $3/4$ también puede escribirse como $0.75$,
y $-12/5$ como $-2.4$. Estas dos formas son equivalentes. En los
racionales, la parte decimal es \textbf{finita} o \textbf{periódica}.

\subsubsection{Aproximación racional a los
irracionales}\label{aproximaciuxf3n-racional-a-los-irracionales}

Por otro lado, existen números reales que no son racionales, en el
sentido de que no existe una razón, fracción o cociente que les sea
igual, pero también pueden escribirse como decimales con una parte
entera y una parte decimal. Estos son los \textbf{irracionales}. Los
irracionales pueden expresarse sintéticamente como el resultado de
alguna operación (como cuando escribimos $\sqrt 2$) o en su forma
decimal. Sin embargo, tienen la característica de que su desarrollo
decimal \textbf{es infinito} no periódico, por lo cual siempre que
escribimos un irracional por su desarrollo decimal, en realidad estamos
\textbf{truncando} ese desarrollo a alguna porción inicial. Jamás
podremos escribir la sucesión completa de decimales.

De manera que, al escribir irracionales en su forma decimal, en realidad
siempre tratamos con \textbf{aproximaciones racionales} a esos
irracionales. Por ejemplo, $3.14$ y $3.1459$ son aproximaciones
racionales al verdadero valor irracional de $PI$, cuya parte decimal
tiene infinitos dígitos.

\subsubsection{Coma o punto decimal}\label{coma-o-punto-decimal}

Al escribir un número con cifras decimales en nuestro sistema numérico
habitual de base 10, usamos una marca especial para separar la parte
entera de la decimal: es la \textbf{coma o punto decimal}. En el
desarrollo decimal, la coma o punto decimal señala el lugar donde los
exponentes de la base en el desarrollo de potencias de 10 \textbf{se
hacen negativos}. Cuando queremos representar números fraccionarios con
computadoras, nos vemos en el problema de representar este signo
especial.

Podemos trasladar la idea de coma o punto decimal al sistema binario. Si
extendemos la Expresión General con exponentes negativos, podemos
escribir números fraccionarios en base 2.

\subsubsection{Fraccionario en base 2 a
decimal}\label{fraccionario-en-base-2-a-decimal}

Si encontramos una expresión como $11.1 0 1 en base 2$, la Expresión General
extendida nos dice cómo obtener su valor en base 10: \[11.1 0 1 en base 2  = \]
\[1  por  2 elevado a la 1 + 1  por  2 elevado a la 0 +\]
\[1  por  2 elevado a la {-1} + 0  por  2 elevado a la {-2} + 1  por  2 elevado a la {-3} =\]
\[2 + 1 + 0.5 + 0 + 0.125 = \] \[3.625\]

\textbf{Otra manera}

Otra manera de obtener el valor decimal de un número fraccionario $n$ en
base 2 consiste en utilizar el hecho de que cada vez que desplazamos el
punto fraccionario un lugar hacia la derecha, estamos multiplicando $n$
por 2, y viceversa, si desplazamos el punto hacia la izquierda, lo
dividimos por 2.

El método consiste en:

\begin{itemize}
\itemsep1pt\parskip0pt\parsep0pt
\item
  Identificar cuántas posiciones fraccionarias tiene $n$ (llamémoslas
  $k$)
\item
  Multiplicar $n$ por $2 elevado a la k$ obteniendo un \textbf{entero} en base 2
\item
  Convertir el entero resultante a base 10, lo cual ya sabemos hacer
\item
  Dividir el resultado por $2 elevado a la k$, obteniendo $n$ en base 10
\end{itemize}

Con este método esencialmente estamos calculando el valor decimal de $n$
\textbf{sin considerar el signo de coma fraccionaria} (es decir,
imaginando que $n$ fuera un entero); convirtiendo ese valor a decimal, y
luego dividiendo el resultado en base 10 por $2 elevado a la k$ para recuperar el
valor original de $n$, sólo que ahora en base 10.

\textbf{Ejemplo}

El número $n = 11.1 0 1 en base 2$ tiene tres cifras decimales ($k = 3$). Lo
convertimos en entero dejando $1 1 1 0 1 en base 2$; averiguamos que este número
en base 10 es 29; y finalmente dividimos 29 por $2 elevado a la 3$. Concluimos que
$n = 11.1 0 1 en base 2 = 29/8 = 3.625$.

\subsubsection{Decimal fraccionario a base
2}\label{decimal-fraccionario-a-base-2}

Para convertir un decimal con parte fraccionaria a base 2:

\begin{enumerate}
\def\labelenumi{\arabic{enumi}.}
\itemsep1pt\parskip0pt\parsep0pt
\item
  Se separan la parte entera (PE) y la parte fraccionaria (PF).
\item
  Se convierte la PE a base 2 separadamente.
\item
  La PF se multiplica por 2 y se toma la PE del resultado. Este dígito
  binario se agrega al resultado.
\item
  Se repite el paso anterior hasta llegar a 0, o hasta lograr la
  precisión deseada.
\end{enumerate}

\textbf{Ejemplo}

Convirtamos el número $n = 3.625$ a base 2. Primero separamos parte
entera (3) y parte fraccionaria (0.625).

\textbf{Parte entera}

La parte entera de $n$ se convierte a base 2 como entero sin signo
(dando $11 en base 2$).

\textbf{Parte fraccionaria}

Para calcular la parte fraccionaria binaria de $n$ seguimos un
procedimiento iterativo (es decir, que consta de pasos que se repiten).

La parte fraccionaria decimal de $n$ se multiplica por 2:
$0.625  por  2 = 1.25$. Separamos este resultado a su vez en parte
entera y parte fraccionaria. Guardamos la parte entera del resultado
(que es 1) \textbf{y repetimos}, es decir, volvemos a multiplicar por 2
la parte fraccionaria recién obtenida (que es 0.25), separamos la parte
entera, etc.

\textbf{La sucesión de dígitos aparecidos como partes enteras} durante
este procedimiento servirán para \textbf{construir la parte
fraccionaria} del resultado. Notemos que estos dígitos que aparecen
solamente pueden ser ceros y unos, porque son la parte entera de
$2 por  x$ con $x < 1$.

El procedimiento de separar, guardar, multiplicar, se repite hasta que
la parte entera obtenida sea 0 (ya no tiene sentido seguir el
procedimiento porque el resultado será siempre 0) o hasta que tengamos
suficientes dígitos computados para nuestra aplicación.

El resultado final es la suma, en base 2, de la parte entera de $n$
calculada anteriormente, más una parte fraccionaria construida con los
dígitos que fueron apareciendo durante el procedimiento de duplicar la
parte fraccionaria.

La conversión a base 2 del número $n = 3.625$ que buscábamos será
$11 en base 2 + 0.1 0 1 en base 2 = 11.1 0 1 en base 2$.

\subsection{Representación de punto
fijo}\label{representaciuxf3n-de-punto-fijo}

¿Cómo aplicamos el método de conversión visto, de fraccionarios
decimales a binarios y viceversa, en las computadoras? El problema es
parecido al de almacenar el signo ``menos'': no podemos guardar en la
memoria otra cosa que bits, de forma que habrá que establecer alguna
convención para indicar dónde está el punto o coma fraccionaria.

A veces las computadoras utilizan sistemas \textbf{de punto fijo} para
representar números con parte fraccionaria. Los sistemas de punto fijo
establecen una cantidad de bits o \textbf{ancho} total (que llamaremos
$n$) y una cantidad fija de bits para la parte fraccionaria (que
llamaremos $k$). Todos los datos manipulados por la computadora tienen
la misma cantidad $n-k$ de bits de parte entera y la misma cantidad $k$
de bits de parte fraccionaria. Por ejemplo, la notación $PF(8,3)$ denota
un \textbf{sistema de punto fijo con 8 bits en total, de los cuales 3
son para la parte fraccionaria}.

Esta convención contiene toda la información necesaria. Al ser fijos los
anchos de parte entera y fraccionaria, la computadora \textbf{puede
tratar aritméticamente a todos los números como si fueran enteros}, sin
preocuparse por partes enteras ni fraccionarias. Solamente habrá que
utilizar la convención al momento de imprimir o comunicar un resultado.
La impresora, o la pantalla, deberán mostrar un resultado con coma
fraccionaria en el lugar correcto.

Sin embargo, todas las operaciones intermedias, entre datos expresados
en punto fijo, habrán podido llevarse a cabo sin tener en cuenta el
lugar de la coma. Dos números en punto fijo se sumarán como si los
representados fueran dos enteros.

\textbf{Ejemplo}

\begin{itemize}
\itemsep1pt\parskip0pt\parsep0pt
\item
  Supongamos que queremos computar $3.625 + 1.25$ en un sistema
  $PF(8,3)$.
\item
  Las conversiones de estos sumandos a fraccionarios binarios son,
  respectivamente, $11.1 0 1$ y $1.01$.
\item
  Pero en la memoria se almacenarán como \textbf{0 0 0 1 1 1 0 1} y
  \textbf{0 0 0 0 1 0 1 0}. Nótese que al ser todas las partes fraccionarias
  del mismo ancho, quedan automáticamente ``encolumnados'' los
  invisibles puntos fraccionarios.
\item
  La suma se efectuará bit a bit como si se tratara de enteros y será
  \textbf{0 0 1 0 0 1 1 1}.
\item
  Si pedimos a la computadora que imprima este valor, aplicará la
  convención $PF(8,3)$ e imprimirá \textbf{0 0 1 0 0.1 1 1}, o su
  interpretación en decimal, $4.875$, que es efectivamente
  $3.625 + 1.25$.
\end{itemize}

\subsubsection{Decimal a PF(n,k)}\label{decimal-a-pfnk}

Para representar un decimal fraccionario $a$, positivo o negativo, en
notación de punto fijo en $n$ lugares con $k$ fraccionarios ($PF(n,k)$),
necesitamos obtener su parte entera y su parte fraccionaria, y expresar
cada una de ellas en la cantidad de bits adecuada a la notación. Para
esto completaremos la parte entera con ceros a la izquierda hasta
obtener $n-k$ dígitos, y completaremos la parte fraccionaria con ceros
por la derecha, hasta obtener $k$ dígitos. Una vez expresado así, lo
tratamos como si en realidad fuera $a  por  2 elevado a la k$, y por lo tanto, un
entero.

\begin{itemize}
\itemsep1pt\parskip0pt\parsep0pt
\item
  Si es positivo, calculamos la secuencia de dígitos binarios que
  expresan su parte entera y su parte fraccionaria, y escribimos ambas
  sobre la cantidad de bits adecuada.
\item
  Si es negativo, consideramos su valor absoluto y procedemos como en el
  punto anterior. Luego complementamos a 2 como si se tratara de un
  entero.
\end{itemize}

\subsubsection{Truncamiento}\label{truncamiento}

Al escribir la parte fraccionaria de un número $a$ en $k$ bits (porque
ésta es la capacidad del sistema de representación de punto fijo con $k$
dígitos fraccionarios), en el caso general estaremos \textbf{truncando}
el desarrollo fraccionario. El número $a$ podría tener otros dígitos
diferentes de cero más allá de la posición $k$. Sin embargo, el sistema
no permite representarlos, y esa información se perderá.

La consecuencia del truncamiento es la aparición de un \textbf{error de
truncamiento} o pérdida de precisión. El número almacenado en el sistema
PF(n,k) será una aproximación con $k$ dígitos fraccionarios al número
original $a$, y no estará representándolo con todos sus dígitos
fraccionarios.

¿Cuál es el valor de este error de truncamiento, es decir, cuál es,
cuantitativamente, la diferencia entre $a$ y la representación en
PF(n,k)? Si los primeros $k$ dígitos del desarrollo fraccionario real de
$a$ se han conservado, entonces la diferencia es menor que $2 elevado a la {-k}$.

\textbf{Ejemplo}

Representemos 3.1459 en notación PF(8,3). Parte entera: 0 0 0 1 1. Parte
fraccionaria: 0 0 1. Representación obtenida: 0 0 0 1 1 0 0 1. Reconvirtiendo
0 0 0 1 1 0 0 1 a decimal, obtenemos parte entera 3 y parte fraccionaria 0.125;
de modo que el número representado en PF(8,3) como 0 0 0 1 1 0 0 1 es en
realidad \textbf{3.125} y no 3.1459.

El error de truncamiento es $3.1459 - 3.1250 = 0.0209$, que es menor que
$2 elevado a la {-3} = 0.125$.

\subsubsection{PF(n,k) a decimal}\label{pfnk-a-decimal}

Para convertir un binario en notación de punto fijo en $n$ lugares con
$k$ fraccionarios (PF(n,k)) a decimal:

\begin{itemize}
\itemsep1pt\parskip0pt\parsep0pt
\item
  Si es positivo, aplicamos la Expresión General extendida, utilizando
  los exponentes negativos para la parte fraccionaria.
\item
  O bien, lo consideramos como un entero, convertimos a decimal y
  finalmente lo dividimos por $2 elevado a la k$.
\item
  Si es negativo, lo complementamos a 2 y terminamos operando como en el
  caso positivo.
\item
  Finalmente agregamos el signo $-$ para expresar que se trata de un
  número negativo.
\end{itemize}

\subsubsection{Preguntas}\label{preguntas}

\begin{itemize}
\itemsep1pt\parskip0pt\parsep0pt
\item
  ¿A qué número decimal corresponde\ldots{}

  \begin{itemize}
  \itemsep1pt\parskip0pt\parsep0pt
  \item
    $0 0 1 1.0 0 0 0$?
  \item
    $0 0 0 1.1 0 0 0$?
  \item
    $0 0 0 0.1 1 0 0$?
  \end{itemize}
\item
  ¿Cómo se representan en $PF(8,4)$\ldots{}

  \begin{itemize}
  \itemsep1pt\parskip0pt\parsep0pt
  \item
    $0.5$?
  \item
    $-7.5$?
  \end{itemize}
\item
  ¿Cuál es el RR de $PF(8,3)$? ¿Y de $PF(8,k)$?
\end{itemize}

La representación de punto fijo es adecuada para cierta clase de
problemas donde los datos que se manejan son de magnitudes y precisiones
comparables.

\begin{itemize}
\itemsep1pt\parskip0pt\parsep0pt
\item
  En la situación contraria, cuando las magnitudes de los datos son muy
  variadas, habrá datos de valor absoluto muy grande, lo que hará que
  sea necesario elegir una representación de una gran cantidad de bits
  de ancho. Pero esta cantidad de bits quedará desperdiciada al
  representar los datos de magnitud pequeña.\\
\item
  Otro tanto ocurre con los bits destinados a la parte fraccionaria. Si
  los requerimientos de precisión de los diferentes datos son muy altos,
  será necesario reservar una gran cantidad de bits para la parte
  fraccionaria. Esto permitirá almacenar los datos con mayor cantidad de
  dígitos fraccionarios, pero esos bits quedarán desperdiciados al
  almacenar otros datos.
\end{itemize}

Las ventajas de la representación en punto fijo provienen, sobre todo,
de que permite reutilizar completamente la lógica ya implementada para
tratar enteros en complemento a 2, sin introducir nuevos problemas ni
necesidad de nuevos recursos. Como la lógica para C2 es sencilla y
rápida, la representación de punto fijo es adecuada para sistemas que
deben ofrecer una determinada \emph{performance}:

\begin{itemize}
\itemsep1pt\parskip0pt\parsep0pt
\item
  Los sistemas que deben ofrecer un tiempo de respuesta corto,
  especialmente aquellos interactivos, como los juegos.
\item
  Los de tiempo real, donde la respuesta a un cómputo debe estar
  disponible en un tiempo menor a un plazo límite, generalmente muy
  corto.
\item
  Los sistemas empotrados o embebidos, que suelen enfrentar
  restricciones de espacio de memoria y de potencia de procesamiento.
\end{itemize}

Por el contrario, algunas clases de programas suelen manipular datos de
otra naturaleza. No es raro que aparezcan en el mismo programa, e
incluso en la misma instrucción de programa, datos o variables de
magnitud o precisión extremadamente diferentes.

Por ejemplo, si un programa de cómputo científico necesita calcular el
\textbf{tiempo en que la luz recorre una millonésima de milímetro}, la
fórmula a aplicar relacionará la velocidad de la luz en metros por
segundo (unos $300.0 0 0.0 0 0 m/s$) con el tamaño en metros de un nanómetro
($0.0 0 0 0 0 0 0 0 1 m$).

Estos dos datos son extremadamente diferentes en magnitud y cantidad de
dígitos fraccionarios. La velocidad de la luz es un número
astronómicamente grande en comparación a la cantidad de metros en un
nanómetro; y la precisión con que necesitamos representar al nanómetro
no es para nada necesaria al representar la velocidad de la luz.

\subsubsection{Notación Científica}\label{notaciuxf3n-cientuxedfica}

En Matemática, la respuesta al problema del cálculo con variables tan
diferentes existe desde hace mucho tiempo, y es la llamada
\textbf{Notación Científica}. En Notación Científica, los números se
expresan en una forma estandarizada que consiste de un
\textbf{coeficiente, significando o mantisa} multiplicado por
\textbf{una potencia de 10}. Es decir, la forma general de la notación
es $m  por  10 elevado a la e$, donde $m$, el coeficiente, \textbf{es un número
positivo o negativo}, y $e$, el \textbf{exponente}, es un entero
positivo o negativo.

La notación científica puede representar entonces números muy pequeños y
muy grandes, todos en el mismo formato, con economía de signos y
permitiendo operar entre ellos con facilidad. Al operar con cantidades
en esta notación podemos aprovechar las reglas del Álgebra para calcular
$m$ y $e$ separadamente, y evitar cuentas con muchos dígitos.

\textbf{Ejemplo}

Los números mencionados hace instantes, la velocidad de la luz en metros
por segundo, y la longitud en metros de un nanómetro, se representarán
en notación científica como $3 por 10 elevado a la 8$ y $1 por 10 elevado a la {-9}$,
respectivamente.

El tiempo en que la luz recorre una millonésima de milímetro se
computará con la fórmula $t = e/v$, con los datos expresados en notación
científica, como:

\[e = 1 por 10 elevado a la {-9}m\] \[v = 3 por 10 elevado a la {8}m/s\]
\[t = e / v = (1  por  10 elevado a la {-9}\ m) / (3  por  10 elevado a la 8\ m/s) = \]
\[t = 1 / 3  por  10 elevado a la {-9-8}\ s =\] \[t = 0.333  por  10 elevado a la {-17}\ s\]

\subsubsection{Normalización}\label{normalizaciuxf3n}

El resultado que hemos obtenido en el ejemplo anterior debe quedar
\textbf{normalizado} llevando el coeficiente $m$ a un valor
\textbf{mayor o igual que 1 y menor que 10}. Si modificamos el
coeficiente al normalizar, para no cambiar el resultado debemos ajustar
el exponente.

\textbf{Ejemplo}

El resultado que obtuvimos anteriormente al computar
$t = 1 / 3  por  10 elevado a la {-9-8}\ s $ fue $0.333  por  10 elevado a la {-17}\ s$. Este
coeficiente $0.333$ no cumple la regla de normalización porque no es
\textbf{mayor o igual que 1}.

\begin{itemize}
\itemsep1pt\parskip0pt\parsep0pt
\item
  Para normalizarlo, lo multiplicamos por 10, convirtiéndolo en $3.33$.
\item
  Para no cambiar el resultado, dividimos todo por 10 afectando el
  exponente, que de -17 pasa a ser -18.
\item
  El resultado queda normalizado como $0.333 por  10 elevado a la {-18}$.
\end{itemize}

\subsubsection{Normalización en base
2}\label{normalizaciuxf3n-en-base-2}

Es perfectamente posible definir una notación científica en otras bases.
En base 2, podemos escribir números con parte fraccionaria en notación
científica normalizada desplazando la coma o punto fraccionario hasta
dejar una parte entera \textbf{igual a 1} (ya que es el único valor
binario que cumple la condición de normalización) y ajustando el
exponente de base 2, de manera de no modificar el resultado.

\textbf{Ejemplos}

\begin{itemize}
\itemsep1pt\parskip0pt\parsep0pt
\item
  $1 0 0.1 1 1 en base 2 = 1.0 0 1 1 1 en base 2  por  2 elevado a la 2$
\item
  $0.0 0 0 1 1 0 1 en base 2 = 1.1 0 1 en base 2  por  2 elevado a la {-4}$
\end{itemize}

\subsection{Representación en Punto
Flotante}\label{representaciuxf3n-en-punto-flotante}

La herramienta matemática de la Notación Científica ha sido adaptada al
dominio de la computación definiendo métodos de \textbf{representación
en punto flotante}. Estos métodos resuelven los problemas de los
sistemas de punto fijo, abandonando la idea de una cantidad fija de bits
para parte entera y parte fraccionaria. En su lugar, inspirándose en la
notación científica, los formatos de punto flotante permiten escribir
números de un gran rango de magnitudes y precisiones en un campo de
tamaño fijo.

Actualmente se utilizan los estándares de cómputo en punto flotante
definidos por la organización de estándares \textbf{IEEE} (Instituto de
Ingeniería Eléctrica y Electrónica, o ``I triple E'').

Estos estándares son dos, llamados \textbf{IEEE 754 en precisión simple
y en precisión doble}.

\begin{itemize}
\itemsep1pt\parskip0pt\parsep0pt
\item
  IEEE 754 precisión simple

  \begin{itemize}
  \itemsep1pt\parskip0pt\parsep0pt
  \item
    Se define sobre un campo de 32 bits
  \item
    Cuenta con \textbf{1 bit de signo}
  \item
    Reserva \textbf{8 bits para el exponente}
  \item
    Reserva \textbf{23 bits para la mantisa}
  \end{itemize}
\item
  IEEE 754 precisión doble

  \begin{itemize}
  \itemsep1pt\parskip0pt\parsep0pt
  \item
    Se define sobre un campo de 64 bits
  \item
    Cuenta con \textbf{1 bit de signo} igual que en precisión simple
  \item
    Reserva \textbf{11 bits para el exponente}
  \item
    Reserva \textbf{52 bits para la mantisa}
  \end{itemize}
\end{itemize}

La definición de los formatos está acompañada por la especificación de
mecanismos de cálculo para usarlos, manejo de errores y otra información
importante.

En el curso utilizaremos siempre el formato de precisión simple.

\subsubsection{Conversión de decimal a punto
flotante}\label{conversiuxf3n-de-decimal-a-punto-flotante}

Para convertir manualmente un número decimal $n$ a punto flotante
necesitamos calcular los tres elementos del formato de punto flotante:
\textbf{signo} (que llamaremos $s$), \textbf{exponente} (que llamaremos
$e$) y \textbf{mantisa} (que llamaremos $m$), en la cantidad de bits
correcta según el formato de precisión simple o doble que utilicemos.

Una vez conocidos $s$, $e$ y $m$, sólo resta escribirlos como secuencias
de bits de la longitud que especifica el formato.

\begin{enumerate}
\def\labelenumi{\arabic{enumi}.}
\itemsep1pt\parskip0pt\parsep0pt
\item
  Separar el \textbf{signo} y escribir el valor absoluto de $n$ en base
  2.

  \begin{itemize}
  \itemsep1pt\parskip0pt\parsep0pt
  \item
    Si $n$ es positivo (respectivamente, negativo), $s$ será 0
    (respectivamente, 1). Separado el signo, consideramos únicamente el
    \textbf{valor absoluto} de $n$ y lo representamos en base 2 como se
    vio al convertir un decimal fraccionario a base 2.
  \end{itemize}
\item
  Escribir el valor binario de $n$ en notación científica \textbf{en
  base 2 normalizada}.

  \begin{itemize}
  \itemsep1pt\parskip0pt\parsep0pt
  \item
    Para convertir $n$ a notación científica lo multiplicamos por una
    potencia de 2 de modo que la parte entera sea 1 (condición para la
    normalización). El resto de la expresión binaria se convierte en
    parte fraccionaria. Para no cambiar el valor de $n$, lo
    multiplicamos por una potencia de 2 inversa a aquella que
    utilizamos.
  \end{itemize}
\item
  El exponente, positivo o negativo, que aplicamos en el paso anterior
  debe ser expresado en notación en exceso a 127.

  \begin{itemize}
  \itemsep1pt\parskip0pt\parsep0pt
  \item
    Al exponente se le suma 127 para representar valores en el intervalo
    $[-127,128]$ con 8 bits. Esta representación se elige para poder
    hacer comparables directamente dos números expresados en punto
    flotante.
  \end{itemize}
\item
  El coeficiente calculado se guarda \textbf{sin su parte entera} en la
  parte de mantisa.

  \begin{itemize}
  \itemsep1pt\parskip0pt\parsep0pt
  \item
    Como la normalización obliga a que la parte entera de la mantisa sea
    1, no tiene mayor sentido utilizar un bit para guardarlo en el
    formato de punto flotante: guardarlo no aportaría ninguna
    información. Por eso basta con almacenar la parte fraccionaria de la
    mantisa, hasta los 23 bits disponibles (o completando con ceros).
  \end{itemize}
\end{enumerate}

\subsubsection{Ejemplo de Punto
Flotante}\label{ejemplo-de-punto-flotante}

Recorramos los pasos para la conversión manual a punto flotante
precisión simple, partiendo del decimal $n = -5.5$. Recordemos que
necesitamos averiguar $s$, $e$ y $m$.

\begin{itemize}
\itemsep1pt\parskip0pt\parsep0pt
\item
  $n$ es negativo, luego $s = 1$.
\item
  $|n| = 5.5$. Convirtiendo el valor absoluto a binario obtenemos
  $1 0 1.1 en base 2$.
\item
  Normalizando, queda $1 0 1.1 en base 2 = 1.0 1 1 en base 2 por  2 elevado a la 2$.
\item
  Del paso anterior, el exponente 2 se representa en exceso a 127 como
  $e = 2 + 127 = 129$. En base 2, $129 = 1 0 0 0 0 0 0 1 en base 2$.
\item
  Del mismo paso anterior extraemos la mantisa quitando la parte entera:
  $1.0 1 1 - 1 = 0.0 1 1$. Los bits de $m$ son $0 1 1 0 0 0 0 0 0...$ con ceros
  hasta la posición 23.
\item
  Finalmente, $s, e, m = 1, 1 0 0 0 0 0 0 1, 0 1 1 0 0 0 0 0 0 0 0 0...$.
\end{itemize}

Lo que significa que la representación en punto flotante de $-5.5$ es
igual a $1 1 0 0 0 0 0 0 1 0 1 1 0 0 0 0 0 0 0...$ (con ceros hasta completar los 32 bits
de ancho total).

\subsubsection{Expresión de punto flotante en
hexadecimal}\label{expresiuxf3n-de-punto-flotante-en-hexadecimal}

Para facilitar la escritura y comprobación de los resultados, es
conveniente leer los 32 bits de la representación en punto flotante
precisión simple como si se tratara de 8 dígitos hexadecimales. Se
aplica la regla, que ya conocemos, de sustituir directamente cada grupo
de 4 bits por un dígito hexadecimal.

Así, en el ejemplo anterior, la conversión del decimal $-5.5$ resultó en
la secuencia de bits $1 1 0 0 0 0 0 0 1 0 1 1 0 0 0 0 0...$ (con más ceros).

Es fácil equivocarse al transcribir este resultado. Pero sustituyendo
los bits, de a grupos de 4, por dígitos hexadecimales, obtenemos la
secuencia equivalente $C0B0 0 0 0 0$, que es más simple de leer y de
comunicar.

\subsubsection{Conversión de punto flotante a
decimal}\label{conversiuxf3n-de-punto-flotante-a-decimal}

Teniendo un número expresado en punto flotante precisión simple,
queremos saber a qué número decimal equivale. Separamos la
representación en sus componentes $s$, $e$ y $m$, que tienen \textbf{1,
8 y 23 bits} respectivamente, y ``deshacemos'' la transformación que
llevó a esos datos a ocupar esos lugares. De cada componente obtendremos
un factor de la fórmula final.

\begin{itemize}
\itemsep1pt\parskip0pt\parsep0pt
\item
  Signo

  \begin{itemize}
  \itemsep1pt\parskip0pt\parsep0pt
  \item
    El valor de $s$ nos dice si el decimal es positivo o negativo.
  \item
    La fórmula $(-1) elevado a la s$ da -1 si $s=1$, y 1 si $s=0$.
  \end{itemize}
\item
  Exponente

  \begin{itemize}
  \itemsep1pt\parskip0pt\parsep0pt
  \item
    El exponente está almacenado en la representación IEEE 754 como ocho
    bits en exceso a 127. Corresponde \textbf{restar 127} para volver a
    obtener el exponente de 2 que afectaba al número originalmente en
    notación científica normalizada.
  \item
    La fórmula $2 elevado a la {(e - 127)}$ dice cuál es la potencia de 2 que debemos
    usar para ajustar la mantisa.
  \end{itemize}
\item
  Mantisa

  \begin{itemize}
  \itemsep1pt\parskip0pt\parsep0pt
  \item
    La mantisa está almacenada sin su parte entera, que en la notación
    científica normalizada en base 2 \textbf{siempre es 1}. Para
    recuperar el coeficiente o mantisa original hay que restituir esa
    parte entera igual a 1.
  \item
    La fórmula $1 + m$ nos da la mantisa binaria original.
  \end{itemize}
\end{itemize}

Reuniendo las fórmulas aplicadas a los tres elementos de la
representación, hacemos el cálculo multiplicando los tres factores:

\[n = (-1) elevado a la s  por  2 elevado a la {(e-127)}   por  (1+m)\]

obteniendo finalmente el valor decimal representado.

\textbf{Ejemplo}

Para el valor de punto flotante IEEE 754 precisión simple representado
por la secuencia hexadecimal $C0B0 0 0 0 0$, encontramos que $s=1$, $e=129$,
$m=0 1 1 0 0 0...$.

\begin{itemize}
\itemsep1pt\parskip0pt\parsep0pt
\item
  Signo

  \begin{itemize}
  \itemsep1pt\parskip0pt\parsep0pt
  \item
    $(-1) elevado a la s = (-1) elevado a la 1 = -1$
  \end{itemize}
\item
  Exponente

  \begin{itemize}
  \itemsep1pt\parskip0pt\parsep0pt
  \item
    $e = 129$ → $2 elevado a la {(e-127)} = 2 elevado a la 2$
  \end{itemize}
\item
  Mantisa

  \begin{itemize}
  \itemsep1pt\parskip0pt\parsep0pt
  \item
    $m = 0 1 1 0 0 0 0...$ → $(1 + m) = 1.0 1 1 0 0 0....$
  \end{itemize}
\end{itemize}

Ajustando la mantisa $1.0 1 1 0 0 0...$ por el factor $2 elevado a la 2$ obtenemos
$1 0 1.1$. Convirtiendo a decimal obtenemos $5.5$. Aplicando el signo
recuperamos finalmente el valor $-5.5$, que es lo que está representando
la secuencia $C0B0 0 0 0 0$.

\subsubsection{Error de truncamiento}\label{error-de-truncamiento}

Aunque los 23 bits de mantisa del formato de punto flotante en precisión
simple son suficientes para la mayoría de las aplicaciones, existen
números que no pueden ser representados, ni aun en doble precisión. El
caso más evidente es el de aquellos números que por su magnitud caen
fuera del rango de representación del sistema. Sin embargo, el formato
IEEE 754 también encuentra limitaciones al tratar con números
aparentemente tan pequeños como 0.1 o 0.2. ¿Cuál es el problema en este
caso?

Si hacemos manualmente el cálculo de la parte fraccionaria binaria de
0.1 (o de 0.2) encontraremos que esta parte fraccionaria es
\textbf{periódica}. Esto ocurre porque $0.1 = 1/10$, y el denominador 10
contiene factores que no dividen a la base (es decir, el 5, que no
divide a 2). Lo mismo ocurre en base 10 cuando computamos $1/3$, que
tiene infinitos decimales periódicos porque el denominador 3 no divide a
10, la base.

Cuando un lenguaje de computación reconoce una cadena de caracteres como
``0.1'', introducida por el programador o el usuario, advierte que se
está haciendo referencia a un número con decimales, e intenta
representarlo en la memoria como un número en punto flotante. La parte
fraccionaria debe ser forzosamente \textbf{truncada}, ya sea a los 23
bits, porque se utiliza precisión simple, o a los 52 bits, cuando se
utiliza precisión doble. En ambos casos, el número representado es una
aproximación al 0.1 original, y esta aproximación será mejor cuantos más
bits se utilicen; pero en cualquier caso, esta parte fraccionaria
almacenada en la representación en punto flotante es \textbf{finita}, de
manera que nunca refleja el verdadero valor que le atribuimos al número
original.

A partir del momento en que ese número queda representado en forma
aproximada, todos los cómputos realizados con esa representación
adolecen de un \textbf{error de truncamiento}, que va agravándose a
medida que se opera con ese número representado.

En precisión simple, se considera que tan sólo \textbf{los primeros
siete decimales} de un número en base 10 son representados en forma
correcta. En precisión doble, sólo los primeros quince decimales son
correctos.

\subsubsection{Casos especiales en punto
flotante}\label{casos-especiales-en-punto-flotante}

En el estándar IEEE 754, no todas las combinaciones de $s$, $e$ y $m$
dan representaciones con sentido, o con el sentido esperable.

Por ejemplo, con las fórmulas presentadas, no es posible representar el
\textbf{cero}, ya que toda mantisa normalizada lleva una parte entera
igual a 1, y los demás factores nunca pueden ser iguales a 0. Entonces,
para representar el 0 en IEEE 754 se recurre a una \textbf{convención},
que se ha definido como la combinación de \textbf{exponente 0 y mantisa
0}, cualquiera sea el signo.

Los números \textbf{normalizados} en IEEE 754 son aquellos que provienen
de una expresión en notación científica normalizada con exponente
diferente de -127, y son la gran mayoría de los representables. Sin
embargo, el estándar permite la representación de una clase de números
muy pequeños, con parte entera 0 en la notación científica, que son los
llamados \textbf{desnormalizados}.

Otros números especiales son aquellos donde el exponente consiste en
ocho \textbf{unos} binarios con mantisa 0. Estos casos están reservados
para representar los valores \textbf{infinito} positivo y negativo (que
aparecen cuando una operación arroja un resultado de \textbf{overflow}
del formato de punto flotante).

Similarmente, cuando el exponente vale ocho unos, y la mantisa es
diferente de 0, se está representando un caso de \textbf{NaN}
(\textbf{Not a Number}, ``no es un número''). Estos casos patológicos
sólo ocurren cuando un proceso de cálculo lleva a una condición de error
(por intentar realizar una operación sin sentido en el campo real, como
obtener una raíz cuadrada de un real negativo).

\end{document}
