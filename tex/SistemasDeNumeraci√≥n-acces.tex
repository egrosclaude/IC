\PassOptionsToPackage{unicode=true}{hyperref} % options for packages loaded elsewhere
\PassOptionsToPackage{hyphens}{url}
%
\documentclass[spanish,a4paper,]{article}
\usepackage[]{sans}
\usepackage{amssymb,amsmath}
\usepackage{ifxetex,ifluatex}
\usepackage{fixltx2e} % provides \textsubscript
\ifnum 0\ifxetex 1\fi\ifluatex 1\fi=0 % if pdftex
  \usepackage[T1]{fontenc}
  \usepackage[utf8]{inputenc}
  \usepackage{textcomp} % provides euro and other symbols
\else % if luatex or xelatex
  \usepackage{unicode-math}
  \defaultfontfeatures{Ligatures=TeX,Scale=MatchLowercase}
\fi
% use upquote if available, for straight quotes in verbatim environments
\IfFileExists{upquote.sty}{\usepackage{upquote}}{}
% use microtype if available
\IfFileExists{microtype.sty}{%
\usepackage[]{microtype}
\UseMicrotypeSet[protrusion]{basicmath} % disable protrusion for tt fonts
}{}
\IfFileExists{parskip.sty}{%
\usepackage{parskip}
}{% else
\setlength{\parindent}{0pt}
\setlength{\parskip}{6pt plus 2pt minus 1pt}
}
\usepackage{hyperref}
\hypersetup{
            pdfborder={0 0 0},
            breaklinks=true}
\urlstyle{same}  % don't use monospace font for urls
\setlength{\emergencystretch}{3em}  % prevent overfull lines
\providecommand{\tightlist}{%
  \setlength{\itemsep}{0pt}\setlength{\parskip}{0pt}}
\setcounter{secnumdepth}{0}
% Redefines (sub)paragraphs to behave more like sections
\ifx\paragraph\undefined\else
\let\oldparagraph\paragraph
\renewcommand{\paragraph}[1]{\oldparagraph{#1}\mbox{}}
\fi
\ifx\subparagraph\undefined\else
\let\oldsubparagraph\subparagraph
\renewcommand{\subparagraph}[1]{\oldsubparagraph{#1}\mbox{}}
\fi

% set default figure placement to htbp
\makeatletter
\def\fps@figure{htbp}
\makeatother

\ifnum 0\ifxetex 1\fi\ifluatex 1\fi=0 % if pdftex
  \usepackage[shorthands=off,main=spanish]{babel}
\else
  % load polyglossia as late as possible as it *could* call bidi if RTL lang (e.g. Hebrew or Arabic)
  \usepackage{polyglossia}
  \setmainlanguage[]{spanish}
\fi

\title{Sistemas de Numeración}
\date{}

\begin{document}
\maketitle

\hypertarget{sistemas-de-numeraciuxf3n}{%
\section{Sistemas de Numeración}\label{sistemas-de-numeraciuxf3n}}

En este primer tema de la unidad veremos las propiedades de los sistemas
de numeración más importantes para el estudio de la arquitectura de
computadoras, en especial los sistemas \textbf{binario y hexadecimal}.

\hypertarget{un-sistema-diferente}{%
\subsection{Un sistema diferente}\label{un-sistema-diferente}}

Todos conocemos el método tradicional de contar con los dedos. Como
tenemos cinco dedos en cada mano, podemos contar hasta diez. Pero
también podemos utilizar un método diferente del tradicional, que
resulta ser muy interesante.

\begin{itemize}
\tightlist
\item
  Con este método, al llegar a 5 con la mano derecha, representamos el 6
  \textbf{sólo con un dedo de la izquierda}. Los dedos de la mano
  derecha \textbf{vuelven a 0}, y seguimos contando con la derecha.
\item
  Cada vez que se agotan los dedos de la mano derecha levantamos un
  nuevo dedo de la izquierda, y la derecha vuelve a 0.
\item
  Cada dedo en alto de la mano izquierda significa que \textbf{se agotó
  la secuencia de la mano derecha una vez}.
\end{itemize}

\hypertarget{preguntas}{%
\subsubsection{Preguntas}\label{preguntas}}

\begin{itemize}
\tightlist
\item
  ¿Hasta qué número se puede representar en este sistema, sólo con dos
  manos?
\item
  Si agregamos una tercera mano, de un amigo, ¿hasta qué número
  llegamos?
\item
  ¿Y cómo se representa el 36? ¿Y el 37?
\item
  Y con cuatro manos, ¿hasta qué número llegamos?
\item
  Y, si el número no se puede representar con dos manos, ¿cómo es el
  procedimiento para saber qué dedos levantar?
\end{itemize}

Notemos que este método tiene mayor capacidad que el tradicional, ya que
podemos contar hasta diez y todavía nos queda mucho por contar con los
dedos de ambas manos.

\hypertarget{sistema-posicional}{%
\subsection{Sistema posicional}\label{sistema-posicional}}

Notemos además que esta ventaja se debe a que el método asigna
\textbf{valores diferentes} a ambas manos. La derecha vale la cantidad
de dedos que muestre, pero la izquierda vale \textbf{seis por su
cantidad de dedos}. Esto se abrevia diciendo que se trata de un
\textbf{sistema de numeración posicional}.

Al tratarse de un sistema posicional, podemos representar números
relativamente grandes con pocos dígitos. En este sistema, disponemos
únicamente de \textbf{6 dígitos (0, 1, 2, 3, 4, 5)} porque ésos son los
que podemos representar con cada mano, es decir, \textbf{en cada
posición}. Pero los números representables solamente dependen de cuántas
manos (o, mejor dicho, de cuántas \textbf{posiciones}) podamos utilizar.

\hypertarget{calculando-cada-posiciuxf3n}{%
\subsubsection{Calculando cada
posición}\label{calculando-cada-posiciuxf3n}}

En este sistema, dado un número no negativo que se pueda representar con
dos manos, podemos saber qué dedos levantar en cada mano haciendo una
sencilla cuenta de división entera (sin decimales): dividimos el número
por 6 y tomamos el cociente y el resto. \textbf{El cociente es el número
de la izquierda, y el resto, el de la derecha}.

Tomemos por ejemplo el número 15. Al dividir 15 por 6, el cociente es 2
y el resto es 3. En este sistema, escribimos el 15 como \textbf{dos
dedos en la izquierda, y tres dedos en la derecha}, lo que podemos
abreviar como \textbf{(2,3)} o directamente \textbf{23} (que se
pronuncia \textbf{dos tres} porque \textbf{no quiere decir veintitrés},
sino \textbf{quince}, sólo que escrito en este sistema). Como el dígito
2 de la izquierda vale por 6, si hacemos la operación de sumar
\textbf{\(2 \times 6 + 3\)} obtenemos, efectivamente, 15.

\hypertarget{base-de-un-sistema-de-numeraciuxf3n}{%
\subsubsection{Base de un sistema de
numeración}\label{base-de-un-sistema-de-numeraciuxf3n}}

La \textbf{base} de un sistema es la cantidad de dígitos de que dispone,
o sea que el sistema decimal habitual es de base 10, mientras que el de
los deditos es de base 6.

\hypertarget{nuxfamero-y-numeral}{%
\subsubsection{Número y numeral}\label{nuxfamero-y-numeral}}

Notemos que un mismo número puede escribirse de muchas maneras: en
prácticamente cualquier base que se nos ocurra, sin necesidad de contar
con los dedos; y que la forma habitual, en base 10, no es más importante
o mejor que las otras (salvo, claro, que ya estamos acostumbrados a
ella). Otras culturas han desarrollado otros sistemas de numeración y
escriben los números de otra manera.

Esto muestra que hay una \textbf{diferencia entre número y numeral},
diferencia que es algo difícil de ver debido a la costumbre de
identificar a los números con su representación en decimal.

\begin{itemize}
\tightlist
\item
  El \textbf{numeral} es lo que escribimos (\(15\),
  \textbf{\(15_{(10}\)} o \textbf{\(23_{(6}\)}).
\item
  El \textbf{número} es la cantidad de la cual estamos hablando (la
  misma en los tres casos).
\end{itemize}

\hypertarget{indicando-la-base}{%
\subsubsection{Indicando la base}\label{indicando-la-base}}

Anteriormente escribíamos \textbf{15} en el sistema de base 6 como
\textbf{23}. Sin embargo, necesitamos evitar la confusión entre ambos
significados de \textbf{23}. Para esto usamos índices subscriptos que
indican la base. Así,

\begin{itemize}
\tightlist
\item
  \textbf{Quince} es \textbf{\(15_{(10}\)} porque está en base diez (la
  del sistema decimal, habitual), y
\item
  \textbf{\(23_{(10}\)} es \textbf{veintitrés},
\item
  pero \textbf{\(23_{(6}\)} es \textbf{dos tres en base 6}, y por lo
  tanto vale \textbf{quince}.
\end{itemize}

Como 10 es nuestra base habitual, cuando no usemos índice subscripto
estaremos sobreentendiendo que hablamos \textbf{en base 10}. Es decir,
\textbf{\(15_{(10}\)} se puede escribir, simplemente, \textbf{\(15\)}.

Cuando queremos pasar un número escrito en una base a un sistema con
otra cantidad de dígitos, el procedimiento de averiguar los dígitos que
van en cada posición se llama \textbf{conversión de base}. Más adelante
veremos procedimientos de conversión de base para cualquier caso que
aparezca.

\hypertarget{sistema-decimal}{%
\subsection{Sistema decimal}\label{sistema-decimal}}

Si reflexionamos sobre el sistema decimal, de diez dígitos, encontramos
que también forma un sistema posicional, sólo que con 10 dígitos en
lugar de los seis del sistema anterior.

Cuando escribimos \textbf{15} en el sistema decimal, esta expresión
equivale a decir ``para saber de qué cantidad estoy hablando, tome el 1
y multiplíquelo por 10, y luego sume el 5''.

Si el número (o, mejor dicho, el \textbf{numeral}) tiene más dígitos,
esos dígitos van multiplicados por \textbf{potencias de 10} que van
creciendo hacia la izquierda. La cifra de las unidades está multiplicada
por la potencia de 10 de exponente 0 (es decir, por \(10^0\), que es
igual a 1).

Esto se cumple para todos los sistemas posicionales, sólo que con
potencias \textbf{de la base correspondiente}.

\hypertarget{sistema-binario}{%
\subsection{Sistema binario}\label{sistema-binario}}

Comprender y manejar la notación en sistema binario es sumamente
importante para el estudio de la computación. El sistema binario
comprende únicamente dos dígitos, \textbf{0 y 1}.

Igual que ocurre con el sistema decimal, los numerales se escriben como
suma de dígitos del sistema multiplicados por potencias de la base. En
este sistema, cada 1 en una posición indica que sumamos una potencia de
2. Esa potencia de 2 es \(2^n\), donde \(n\) es la posición, y las
posiciones se cuentan desde 0.

Por ejemplo,

\[10 = 1\times8 + 0\times4 + 1\times2 + 0\times1 = 1010_{(2}\]

y

\[15 = 1\times8 + 1\times4 + 1\times2 + 1\times1 = 1111_{(2}\]

\hypertarget{trucos-para-conversiuxf3n-ruxe1pida}{%
\subsubsection{Trucos para conversión
rápida}\label{trucos-para-conversiuxf3n-ruxe1pida}}

Las computadoras digitales, tal como las conocemos hoy, almacenan todos
sus datos en forma de números binarios. Es \textbf{muy recomendable},
para la práctica de esta materia, adquirir velocidad y seguridad en la
conversión de y a sistema binario.

Una manera de facilitar esto es memorizar los valores de algunas
potencias iniciales de la base 2:

\[2^0 = 1\] \[2^1 = 2\] \[2^2 = 4\] \[2^3 = 8\] \[2^4 = 16\]
\[2^5 = 32\] \[2^6 = 64\] \[2^7 = 128\]

¿Qué utilidad tiene memorizar esta tabla? Que nos permite convertir
mentalmente algunos casos simples de números en sistema decimal, a base
2. Por ejemplo, el número \textbf{40} equivale a \textbf{32 + 8}, que
son ambas potencias de 2. Luego, la expresión de 40 en sistema binario
será \textbf{101000}.

Otro truco interesante consiste en ver que si un numeral está en base 2,
\textbf{multiplicarlo por 2 equivale a desplazar un lugar a la izquierda
todos sus dígitos, completando con un 0 al final}. Así, si sabemos que
\(40_{(10} = 101000_{(2}\), ¿cómo escribimos rápidamente \textbf{80},
que es \(40\times2\)? Tomamos la expresión de 40 en base 2 y la
desplazamos a la izquierda agregando un 0: \(1010000_{(2} = 80_{(10}\).

De todas maneras, estos no son más que trucos que pueden servir en no
todos los casos. Más adelante veremos el procedimiento de conversión
general correcto.

\textbf{Preguntas}

\begin{itemize}
\tightlist
\item
  ¿Cuál es el truco para calcular rápidamente la expresión binaria de
  20, si conocemos la de 40?
\item
  ¿Cómo calculamos la de 40, si conocemos la de 10?
\item
  ¿Cómo podemos expresar estas reglas en forma general?
\end{itemize}

\hypertarget{sistema-hexadecimal}{%
\subsection{Sistema hexadecimal}\label{sistema-hexadecimal}}

Otro sistema de numeración importante es el hexadecimal o de base 16. En
este sistema tenemos \textbf{más dígitos} que en el decimal, por lo cual
tenemos que recurrir a ``dígitos'' nuevos, tomados del alfabeto. Así, A
representa el 10, B el 11, etc.

El sistema hexadecimal nos resultará útil porque con él podremos
expresar fácilmente números que llevarían muchos dígitos en sistema
binario.

\begin{itemize}
\tightlist
\item
  La conversión entre binarios y hexadecimales es sumamente directa.
\item
  Al ser un sistema con más dígitos que el binario, la expresión de
  cualquier número será más corta.
\end{itemize}

\hypertarget{una-expresiuxf3n-general}{%
\subsection{Una expresión general}\label{una-expresiuxf3n-general}}

Como hemos visto intuitivamente en el sistema de contar con los dedos, y
como hemos confirmado repasando los sistemas decimal, binario y
hexadecimal, los sistemas posicionales tienen una cosa muy importante en
común: las cifras de un \textbf{numeral} escrito en cualquier base no
son otra cosa que los \textbf{factores por los cuales hay que
multiplicar las sucesivas potencias de la base} para saber a qué
\textbf{número} nos estamos refiriendo.

Por ejemplo, el numeral \textbf{2017} escrito en base 10 no es otra cosa
que la suma de:

\[2 \times 1000 + 0 \times 100 + 1 \times 10 + 7 \times 1 = \]
\[2 \times 10^3 + 0 \times 10^2 + 1 \times 10^1 + 7 \times 10^0\]

Los dígitos 2, 0, 1 y 7 multiplican, respectivamente, a \(10^3\),
\(10^2\), \(10^1\) y \(10^0\), que son potencias de la base 10. Este
\textbf{numeral} designa al \textbf{número} 2017 porque esta cuenta,
efectivamente, da \textbf{2017}.

Sin embargo, si el número está expresado en otra base, la cuenta debe
hacerse con potencias de esa otra base. Si hablamos de \(2017_{(8}\),
entonces las cifras 2, 0, 1 y 7 multiplican a \(8^3\), \(8^2\), \(8^1\)
y \(8^0\). Este \textbf{numeral} designa al \textbf{número} 1039 porque
esta cuenta, efectivamente, da \textbf{1039}.

Este análisis permite enunciar una ley o expresión general que indica
cómo se escribe un número \(n\) cualquiera, no negativo, en una base
\(b\):
\[n = x_{k} \times b^k + \ldots + x_{2} \times b^{2} + x_1 \times b^1 +x_{0} \times b^0\]

Esta ecuación puede escribirse más sintéticamente en notación de
sumatoria como:

\[n = \sum_{i=0}^{k}{x_i \times b^i}\]

En estas ecuaciones (que son equivalentes):

\begin{itemize}
\tightlist
\item
  Los números \(x_i\) son las cifras del numeral.
\item
  Los números \(b^i\) son potencias de la base, cuyos exponentes crecen
  de derecha a izquierda y comienzan por 0.
\item
  Las potencias están \textbf{ordenadas y completas}, y son tantas como
  las cifras del numeral.
\item
  Los números \(x_i\) son necesariamente \textbf{menores que \(b\)}, ya
  que son dígitos en un sistema de numeración que tiene \(b\) dígitos.
\end{itemize}

\hypertarget{conversiuxf3n-de-base}{%
\subsection{Conversión de base}\label{conversiuxf3n-de-base}}

Veremos algunos casos interesantes de conversiones de base. Serán
especialmente importantes los casos donde el número de origen o de
destino de la conversión esté en base 10, nuestro sistema habitual, pero
también nos dedicaremos a algunas conversiones de base donde ninguna de
ellas sea 10.

\hypertarget{conversiuxf3n-de-base-10-a-otras-bases}{%
\subsubsection{Conversión de base 10 a otras
bases}\label{conversiuxf3n-de-base-10-a-otras-bases}}

El procedimiento para convertir un número escrito en base 10 a cualquier
otra base (llamémosla \textbf{base destino}) es siempre el mismo y se
basa en la división entera (sin decimales):

\begin{itemize}
\tightlist
\item
  Dividir el número original por la base destino, anotando cociente y
  resto
\item
  Mientras se pueda seguir dividiendo:

  \begin{itemize}
  \tightlist
  \item
    Volver al paso anterior reemplazando el número original por el nuevo
    cociente
  \end{itemize}
\item
  Finalmente escribimos los dígitos de nuestro número convertido usando
  \textbf{el último cociente y todos los restos en orden inverso a como
  aparecieron}.
\end{itemize}

Ésta es la expresión de nuestro número original en la base destino.

\begin{itemize}
\tightlist
\item
  Notemos que cada uno de los restos obtenidos es con toda seguridad
  \textbf{menor que la base destino}, ya que, en otro caso, podríamos
  haber seguido adelante con la división entera.
\item
  Notemos también que el último cociente es también \textbf{menor que la
  base destino}, por el mismo motivo de antes (podríamos haber
  proseguido la división).
\item
  Lo que acabamos de decir garantiza que tanto el último cociente, como
  todos los restos aparecidos en el proceso, \textbf{pueden ser dígitos
  de un sistema en la base destino} al ser todos menores que ella.
\end{itemize}

\textbf{Pregunta}

¿Cómo podemos usar la Expresión General para explicar por qué este
procedimiento es correcto, al menos para el caso de convertir \textbf{61
a base 3}?

\hypertarget{conversiuxf3n-de-otras-bases-a-base-10}{%
\subsubsection{Conversión de otras bases a base
10}\label{conversiuxf3n-de-otras-bases-a-base-10}}

La conversión en el sentido opuesto, de una base \(b\) cualquiera a base
10, se realiza simplemente aplicando la Expresión General. Cada uno de
los dígitos del número original (ahora en base \(b\) arbitraria) es el
coeficiente de alguna potencia de la base original. Esta potencia
depende de la posición de dicho dígito. Una vez que escribimos todos los
productos de los dígitos originales por las potencias de la base,
hacemos la suma y nos queda el resultado: el número original convertido
a base 10.

Es de la mayor importancia cuidar de que las potencias de la base que
intervienen en el cálculo estén \textbf{ordenadas y completas}. Es fácil
si escribimos estas potencias a partir de la derecha, comenzando por la
que tiene exponente 0, y vamos completando los términos de derecha a
izquierda hasta agotar las posiciones del número original.

\hypertarget{preguntas-1}{%
\subsection{Preguntas}\label{preguntas-1}}

¿Cómo sería un sistema de \textbf{contar con los dedos en base 2}? Dedo
arriba = 1, dedo abajo = 0\ldots{}

\begin{itemize}
\tightlist
\item
  ¿Cómo hacemos el 1, el 2, el 3\ldots{}?
\item
  ¿Hasta qué número podemos contar con una mano? ¿Y con dos manos?
\item
  ¿Y cómo se indica el \textbf{4} en este sistema?
\end{itemize}

\hypertarget{conversiuxf3n-entre-bases-arbitrarias}{%
\subsection{Conversión entre bases
arbitrarias}\label{conversiuxf3n-entre-bases-arbitrarias}}

Hemos visto los casos de conversión entre base 10 y otras bases, en
ambos sentidos. Ahora veamos los casos donde ninguna de las bases origen
o destino son la base 10.

La buena noticia es que, en general, \textbf{esto ya sabemos hacerlo}.
Si tenemos dos bases \(b_1\) y \(b_2\) cualesquiera, ninguna de las
cuales es 10, sabiendo hacer las conversiones anteriores podemos hacer
la conversión de \(b_1\) a \(b_2\) sencillamente haciendo \textbf{dos
conversiones pasando por la base 10}. Si queremos convertir de \(b_1\) a
\(b_2\), convertimos primero \textbf{de \(b_1\) a base 10}, aplicando el
procedimiento ya visto, y luego \textbf{de base 10 a \(b_2\)}. Eso es
todo.

Pero en algunos casos especiales podemos aprovechar cierta relación
existente entre las bases a convertir: por ejemplo, cuando son \textbf{2
y 16}, o \textbf{2 y 8}. La base 2 es la del sistema \textbf{binario}, y
las bases 16 y 8 son las del sistema \textbf{hexadecimal} y del sistema
\textbf{octal} respectivamente.

En estos casos, como 16 y 8 son potencias de 2 (la otra base), podemos
aplicar un truco matemático para hacer la conversión en un solo paso y
con muchísima facilidad. Por fortuna son estos casos especiales los que
se presentan con mayor frecuencia en nuestra disciplina.

\hypertarget{equivalencias-entre-sistemas}{%
\subsection{Equivalencias entre
sistemas}\label{equivalencias-entre-sistemas}}

Para poder aplicar este truco se necesita la tabla de equivalencias
entre los dígitos de los diferentes sistemas. Si no logramos
memorizarla, conviene al menos saber reproducirla, asegurándose de saber
\textbf{contar} en las bases 2, 8 y 16 para reconstruir la tabla si es
necesario. Pero con la práctica, se logra memorizarla fácilmente.

Notemos que:

\begin{itemize}
\tightlist
\item
  El sistema octal tiene ocho dígitos \textbf{(0 \ldots{} 7)} y cada uno
  de ellos se puede representar con \textbf{tres dígitos binarios}:

  \begin{itemize}
  \tightlist
  \item
    000
  \item
    001
  \item
    010
  \item
    011
  \item
    100
  \item
    101
  \item
    110
  \item
    111
  \end{itemize}
\end{itemize}

Notemos que:

\begin{itemize}
\tightlist
\item
  El sistema hexadecimal tiene dieciséis dígitos \textbf{(0 \ldots{} F)}
  y cada uno de ellos se puede representar con \textbf{cuatro dígitos
  binarios}:

  \begin{itemize}
  \tightlist
  \item
    0000
  \item
    0001
  \item
    0010
  \item
    0011
  \item
    0100
  \item
    0101
  \item
    0110
  \item
    0111
  \item
    1000
  \item
    1001
  \item
    1010
  \item
    1011
  \item
    1100
  \item
    1101
  \item
    1110
  \item
    1111
  \end{itemize}
\end{itemize}

\hypertarget{conversiuxf3n-entre-sistemas-binario-y-hexadecimal}{%
\subsubsection{Conversión entre sistemas binario y
hexadecimal}\label{conversiuxf3n-entre-sistemas-binario-y-hexadecimal}}

El truco para convertir de base 2 a base 16 consiste simplemente en
agrupar los dígitos binarios de a cuatro, y reemplazar cada grupo de
cuatro dígitos por su equivalente en base 16 según la tabla anterior.

Si hace falta completar un grupo de cuatro dígitos binarios, se completa
con ceros a la izquierda.

Si el problema es convertir, inversamente, de base 16 a base 2, de igual
forma reemplazamos cada dígito hexadecimal por los cuatro dígitos
binarios que lo representan.

\hypertarget{conversiuxf3n-entre-sistemas-binario-y-octal}{%
\subsubsection{Conversión entre sistemas binario y
octal}\label{conversiuxf3n-entre-sistemas-binario-y-octal}}

El problema de convertir entre bases 2 y 8 es igual de sencillo. Basta
con reemplazar cada grupo de \textbf{tres} dígitos binarios (completando
con ceros a la izquierda si hace falta) por el dígito octal equivalente.
Lo mismo si la conversión es en el otro sentido.

\end{document}
